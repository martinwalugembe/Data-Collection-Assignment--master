\documentclass[options]{article}

 \usepackage[
    top    = 2.75cm,
    bottom = 2.50cm,
    left   = 4.00cm,
    right  = 3.50cm]{geometry}

\usepackage[parfill]{parskip}

\title{Development of an advertising form to improve on the bureaucracy delay and inconsistency in advert booking in media houses.}
\author{Walugembe Martin Alvin (216005016, 16/U/12262/EVE) \thanks{supervisor: Ernest Mwebaze}}
\date{%
    Makerere University\\%
    Feb 23, 2018
}


\begin{document}
\begin{titlepage}
\maketitle
\end{titlepage}








\section{\textbf{ Introduction}} 
Most media houses in Uganda are having various products they sell to the public for example newspaper adverts, television adverts as well as radio adverts which has improved their marketability and rate of return on investment.
The basic problem that is currently being faced is the issue of automation of the booking process of these adverts to see to it that there is certainity and accuracy in the booking process and assurance to the clients that their adverts are booked in time. In this regard, the vast use of paper work needs to be checked.\bigbreak
 
\subsection{\textbf{Research Background}}
Due to the vast use of paperwork, various media houses have engaged their procurement and ICT Departments to see to it that the problem is checked.
Through that engagement, ustomer care advisors and assistants have been hired to help out in the follow up of these adverts.
 

 \bigbreak



\subsection{\textbf{Problem Statement}}
Development of an advertising form in various media houses is very important to the clients and staff that are participants in the booking process to avoid the delays, wrong bookings and misappropriate timing in the advert booking process.

\subsection{\textbf{Objectives}}


\subsubsection{\textbf{Main Objective}} 
 To come up with an advertising form that can capture the different types of adverts alongside their dates of run and time in their different categories i.e radio, television and newspapers.

\subsubsection{\textbf{Specific Objectives}}

\begin{itemize}
\item To ensure accuracy in the category of advert being booked
\item To see to it that the adverts are booked in time
  \item To ensure that the adverts booked have run.
  \item To ensure that the form can be accessed by everyone simply
\end{itemize}


\subsection{\textbf{Scope}}
This research is for persons that are actively involved in media or broadcast advertising.

\subsection{\textbf{Purpose of the Study}}
The purpose of this study is to determine the most convenient method in receiving adverts from clients, capturing details by the booking clerks and publishing of the approved artworks on the various media platforms i.e radio, television and newspapers.


\section{\textbf{Methodology}}
This study was made in the best media house in Uganda (\textbf{Vision Group}). An advertising form was developed to provide a well streamlined mode of execution of booking activities of the various media products with respect to their platforms. In addition, the date of run is indicated which can be followed by the booking clerks as they book the adverts as well as the clients follow up to find out whether his/her advert was published. This was accomplished by dedicating a 20 customer care staff to the advertising form cause in case of any issues that arose and to guide the clients as well as fellow staff members  in understanding the form fr efficient service delivery. These were selected by multi-stage random cluster sampling in the Support staff department.

\section{\textbf{Results}}
The designed advertising form streamlined  the status quo of bureaucracy, delay and inconsistency in the advert bookings for the various media houses.We also accessed information about the most used advertising platform, the fastest spreading media coverage category, the most expensive media platform as well as the least costly, what time it takes the published newspapers to get to the market, the media platforms that are listened to most in the different regions of the country among others.

\section{\textbf{Conclusion}}
According to the obtained results, the designed advertising form is capable to improve the advertising process both on the staff and client side because of its provision of the type of adverts, category, date of run and classification of the platform. 
\begin{thebibliography}{10} \bibitem{latexGuide}NewVision Magazine 2017 article , \emph{NVPPCL(September 2017)} \end{thebibliography}



\end{document}